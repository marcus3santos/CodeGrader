% Created 2025-03-12 Wed 16:43
% Intended LaTeX compiler: pdflatex
\documentclass[11pt]{article}
\usepackage[utf8]{inputenc}
\usepackage[T1]{fontenc}
\usepackage{graphicx}
\usepackage{longtable}
\usepackage{wrapfig}
\usepackage{rotating}
\usepackage[normalem]{ulem}
\usepackage{amsmath}
\usepackage{amssymb}
\usepackage{capt-of}
\usepackage{hyperref}
\date{}
\title{Common Lisp Practicum Test}
\hypersetup{
 pdfauthor={marcus},
 pdftitle={Common Lisp Practicum Test},
 pdfkeywords={},
 pdfsubject={},
 pdfcreator={Emacs 30.1 (Org mode 9.7.11)}, 
 pdflang={English}}
\begin{document}

\maketitle
\section*{Questions}
\label{sec:org379d720}

\subsection*{Question 1}
\label{sec:orgf86c001}

\textbf{NOTE}:
\begin{itemize}
\item You are required to write the solutions for the parts of this question in the Lisp program file \textbf{\textasciitilde{}/pt/q1.lisp}
\item You may create helper functions in your program file.
\item You must not use or refer to the following Lisp built-in function(s) and symbol(s): \textbf{COUNT}, \textbf{MEMBER}. The penalty for doing so is a deduction of 80\% on the score of your solutions for this question.
\end{itemize}


\textbf{\textbf{Part 1:}} Write a function \texttt{count-occurrences} that takes an element
and a list as arguments and returns the number of times the element
appears in the list. 

\begin{verbatim}
CL-USER> (COUNT-OCCURRENCES 3 '(1 2 3 3 3 4))
3
CL-USER> (COUNT-OCCURRENCES 'A '(A B A C A))
3
CL-USER> (COUNT-OCCURRENCES 5 '(1 2 3 4))
0
\end{verbatim}


\textbf{\textbf{Part 2:}} Write a function \texttt{contains-all?} that takes two lists as
arguments and returns \texttt{T} if all elements of the first list are
contained in the second list, and \texttt{NIL} otherwise.

\begin{verbatim}
CL-USER> (CONTAINS-ALL? '(1 2) '(1 2 3 4))
T
CL-USER> (CONTAINS-ALL? '(1 5) '(1 2 3 4))
NIL
CL-USER> (CONTAINS-ALL? 'NIL '(1 2 3))
T
\end{verbatim}
\subsection*{Question 2}
\label{sec:org506aedb}

\textbf{NOTE}:
\begin{itemize}
\item You are required to write the solutions for the parts of this question in the Lisp program file \textbf{\textasciitilde{}/pt/q2.lisp}
\item You may create helper functions in your program file.
\item There are no restrictions in the use of Lisp built-in functions or symbols in the parts of this question.
\end{itemize}

\textbf{\textbf{Part 1:}} Write a function \texttt{reverse-list} that takes a list as an
argument and returns a new list that is the reverse of the original
list. 

\begin{verbatim}
CL-USER> (REVERSE-LIST '(1 2 3 4))
(4 3 2 1)
CL-USER> (REVERSE-LIST '(A B C))
(C B A)
CL-USER> (REVERSE-LIST 'NIL)
()
\end{verbatim}

\textbf{\textbf{Part 2:}} Write a function \texttt{palindrome?} that takes a list as an
argument and returns \texttt{T} if the list is a palindrome (reads the same
forwards and backwards), and \texttt{NIL} otherwise. You may not use
\texttt{REVERSE} or \texttt{NREVERSE}.

\begin{verbatim}
CL-USER> (PALINDROME? '(1 2 3 2 1))
T
CL-USER> (PALINDROME? '(A B C D))
NIL
CL-USER> (PALINDROME? 'NIL)
T
\end{verbatim}
\end{document}
